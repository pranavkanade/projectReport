




\begin{center}
\pagestyle{empty}          

\begin{center}				

\begin{LARGE}
\section*{Prologue}
\addcontentsline{toc}{section}{Acknowledgement}
%\vspace{.15 in}       
\end{LARGE}

\end{center}		


\begin{normalsize}

{\setlength{\baselineskip}{1.2\baselineskip}

\begin{quote}

Project work is one of the most important components of the curriculum for the
Engineering Graduate. From conceiving the idea to the materialization of it is a journey
that has to be systematized, well defined and well documented to enjoy the full benefits
of the efforts undertaken.
Every activity of the project development has its own importance and typical
activities are like: Team formation, conceiving the idea, preparing the hypothesis,
reporting the progress / development to the guide/ mentor, Interactions, suggestions
and improvements, relevant documentations in proper format, schedule plans and visit
logs.
Every institute is following their own best methods and techniques as per the
guidelines and curriculum at the affiliated university. To bring the uniformity and
standardization for the project work there is a need to come together and prepare the
comprehensive guidelines regarding it.
This work book for the project work will serve the purpose and facilitate the job
of students, guide and project coordinator. This document will reflect accountability,
punctuality, technical writing ability and work flow of the work undertaken.
This document will definitely support the work undertaken.
\vspace*{3\baselineskip} \\
\begin{tabular}{p{8.2cm}c}
&Student Name1\\
&Student Name2\\
&Student Name3\\
&Student Name4\\
&(B.E. Computer Engg.)
%}
\end{tabular}





\end{quote}
}
\end{normalsize}
\end{center}































Dr. Varsha H Patil
Coordinator, BoS Computer Engineering
SPPU, Pune